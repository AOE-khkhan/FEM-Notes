\documentclass[12pt,oneside,openany, english]{book}
\usepackage{babel}
\usepackage{babelbib}
\usepackage{etoolbox}
\usepackage{mathtools}
\usepackage{cancel}
\usepackage{amstext, amssymb, amsthm, amsmath, amsbsy}
\usepackage{tabularx}
\usepackage[ansinew]{inputenc}
\usepackage[margin=20pt,font=small,labelfont=bf,labelsep=period]{caption} % Modify caption format
\usepackage{graphicx}
\usepackage{url}
\usepackage{geometry}
\usepackage{enumerate}
\usepackage{subcaption}  %% To include subgraphics
                         %% subfig is deprecated
\usepackage{float}
\usepackage{multirow}
\usepackage{leftidx}
\usepackage[ruled]{algorithm2e} % Include algorithms
\usepackage{listings} % To include code
\usepackage{cite}  % Pretty citations
\usepackage{bookmark} % Bookmarks in PDF
\usepackage{cleveref}
\usepackage{hyperref} % Nice details for pdf
\usepackage[colorinlistoftodos]{todonotes}
\usepackage{physics} % Extra operators and cool stuff
\usepackage{minted}  % Include fancy code
\usepackage{xcolor}
\usepackage[nottoc]{tocbibind} % Add the references to the table of contents
\usepackage{scalerel}  % To create the assembly operator

%% Packages setup

% For babelbib
\patchcmd\btxselectlanguage{\csname}{\csname TEMPPATCH}{}{} 

\geometry{
	verbose,
	letterpaper
}

\hypersetup{
	bookmarksopen=true,
	colorlinks=true,
	linkcolor=blue,
	citecolor=blue,
	urlcolor=black,	
	linktoc=all,
	bookmarksopen=false,
	pdftitle={Finite Element Methods},
	pdfauthor={J. Gomez, N. Guarin-Zapata},
	pdfkeywords={Computational Mechanics, Finite Element Methods},
	pdfsubject={Class Notes},
	pdfpagemode=UseOutlines,
	pdfstartview=FitH
}


\definecolor{bg}{rgb}{0.95,0.95,0.95}

%%%%%%%%%%%%%%%%%%%%%%%%%%%%%%%%%%%%%%%%%%%%%%%%%%%%%%%%%%%%%%%%%%%%%%%%%%%%
% New  commands



\newcommand{\urlbib}[1]{{\footnotesize{\url{#1}}}} % URL in references
\newcommand{\fullref}[1]{\ref{#1} at page \pageref{#1}}

% make "C++" look pretty when used in text by touching up the plus signs
\newcommand{\CPP}
{C\nolinebreak[4]\hspace{-.05em}\raisebox{.22ex}{\footnotesize\bf ++}}

% Assembly operator
%\newcommand{\assem}[2]{
%	\overset{#2}{\underset{#1}{
%		\mathbb{A}}
%	}
%}

\DeclareMathOperator*{\assem}{\scalerel*{\mathbb{A}}{\sum‌​}} 

%% Example:
% \[\assem_{k=1}^{n} N_k(x)u_k\]

%%--- Math commands ---


% Integrals
\newcommand{\intL}{\int\limits}


%%%%%%%%%%%%%%%%%%%%%%%%%%%%%%%%%%%%%%%%%%%%%%%%%%%%%%%%%%%%%%%%%%%%%%%%%%%%
% Tweaks

\setlength{\parskip}{0.5cm}

\crefname{algocf}{alg.}{algs.}
\Crefname{algocf}{Algorithm}{Algorithms}