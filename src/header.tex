\documentclass[12pt,oneside,openany]{book}
\usepackage[english]{babel}
\usepackage{babelbib}
\usepackage{etoolbox}
\usepackage{mathtools}
\usepackage{cancel}
\usepackage{amstext, amssymb, amsthm, amsmath, amsbsy}
\usepackage{tabularx}
\usepackage[ansinew]{inputenc}
\usepackage[margin=20pt,font=small,labelfont=bf,labelsep=period]{caption} % Modify caption format
\usepackage{graphicx}
\usepackage{url}
\usepackage{geometry}
\usepackage{enumerate}
\usepackage{subcaption}  %% To include subgraphics
                         %% subfig is deprecated
\usepackage{float}
\usepackage{multirow}
\usepackage{leftidx}
\usepackage[ruled]{algorithm2e} % Include algorithms
\usepackage{listings} % To include code
%\usepackage{algorithmic}
\usepackage{cite}  % Pretty citations
\usepackage{bookmark} % Bookmarks in PDF
\usepackage{cleveref}
\usepackage{hyperref} % Nice details for pdf
\usepackage[colorinlistoftodos]{todonotes}

%% Packages setup

\patchcmd\btxselectlanguage{\csname}{\csname TEMPPATCH}{}{} % For babelbib

\geometry{
	verbose,
	letterpaper,
	tmargin=3cm,
	bmargin=2cm,
	lmargin=2cm,
	rmargin=2cm
}

\hypersetup{
	bookmarksopen=true,
	colorlinks=true,
	linkcolor=blue,
	citecolor=blue,
	urlcolor=black,	
	linktoc=all,
	pdftitle={Finite Element Methods},
	pdfauthor={J. Gomez, N. Guarin-Zapata},
	pdfkeywords={Computational Mechanics, Finite Element Methods},
	pdfsubject={Class Notes},
	pdfpagemode=UseOutlines,
	pdfstartview=FitH
}
	

%%%%%%%%%%%%%%%%%%%%%%%%%%%%%%%%%%%%%%%%%%%%%%%%%%%%%%%%%%%%%%%%%%%%%%%%%%%%
% New  commands

\setlength{\parskip}{0.5cm}

% Quotation marks:   ``''

\newcommand{\urlbib}[1]{{\footnotesize{\url{#1}}}} % URL in references
\newcommand{\fullref}[1]{\ref{#1} at page \pageref{#1}}


% Partial derivatives
\newcommand{\pder}[2][]{\frac{\partial #1}{\partial #2}}
\newcommand{\pdern}[3][]{\frac{\partial^{#3} #1}{\partial #2^{#3}}}

%%%%%%%%%%%%%%%%%%%%%%%%%%%%%%%%%%%%%%%%%%%%%%%%%%%%%%%%%%%%%%%%%%%%%%%%%%%%

%%
%\renewcommand{\tablename}{Tabla}
%\renewcommand{\figurename}{Figura}
%\renewcommand{\contentsname}{Table of Contents}
%\renewcommand{\listtablename}{Lista de tablas}
%\renewcommand{\listfigurename}{Lista de figuras}
